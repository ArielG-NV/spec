\Chapter{Semantic Checking}{check}

\Section{Contexts}{context}

A \SpecDef{context} is an ordered sequence of zero or more \SpecDef{context entries}.
Context entries include \SpecDef{bindings} of the name of a variable (an identifier) to its type.

The notation:
\begin{center}
\ContextContains{\ContextVarA}{E}
\end{center}
indicates that the context \ContextVarA\ contains an entry matching $E$.

The notation:
\begin{center}
\ContextLookup{\ContextVarA}{k}{v}
\end{center}
indicates that the right-most entry in context \ContextVarA\ that has key $k$ maps to $v$.

\Section{Expressions}{expr}

Slang use a bidirectional type-checking algorithm.
There are two different judgements used when checking expressions.

\SubSection{Synthesis}{synth}

The notation:
\begin{center}
\SynthExpr{\ContextVarA}{e}{t}{\ContextVarB}
\end{center}
means that under input context \ContextVarA, the expression $e$ \SpecDef{synthesizes} the type $t$, an yields output context \ContextVarB.

\begin{Description}
Synthesis judgements are used in places where an expression needs to be checked, but the expected type of the expression is not known.
In terms of implementation, \ContextVarA and $e$ are inputs, while $t$ and \ContextVarB are outputs.
\end{Description}

\SubSection{Checking}{check}

The notation:
\begin{center}
\CheckExpr{\ContextVarA}{e}{t}{\ContextVarB}
\end{center}
means that under input context \ContextVarA, the expression $e$ \SpecDef{checks} against the type $t$, and yields output context \ContextVarB.

\begin{Description}
Checking judgements are used in places where the type that an expression is expected to have is known.
In terms of implementation, \ContextVarA, $e$, and $t$ are all inputs, and \ContextVarB is the only output.
\end{Description}


\Section{Statements}{stmt}

The notation:
\begin{center}
\CheckStmt{\ContextVarA}{s}{\ContextVarB}
\end{center}
means that under input context \ContextVarA, checking of statement $s$ is successful, and yields output context \ContextVarB.

\Section{Declaration}{decl}

The notation:
\begin{center}
\CheckDecl{\ContextVarA}{d}{\ContextVarB}
\end{center}
means that under input context \ContextVarA, checking of declaration $d$ is successful, and yields output context \ContextVarB.

