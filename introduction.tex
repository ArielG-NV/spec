\Chapter{Introduction}{intro}

\Section{Scope}{scope}

This document is a specification for the Slang programming language.

Slang is language primarily designed for use in *shader programming*: performance oriented GPU programming for real-time graphics.

\Section{Conformance}{conformance}

This document aspires to specify the Slang language, and the behaviors expected of conforming implementations, with a high level of rigor.

The open-source Slang compiler *implementation* may deviate from the language as specified here, in a few key ways:

\begin{itemize}
    \item{The implementation is necessarily imperfect, and can have bugs}
    \item{The implementation may not fully support constructs specified here, or their capabilities may not be as complete as what is required by the specification}
    \item{The implementation may support certain constructs that are experimental, deprecated, or are otherwise intentionally unspecified (or even undocumented)}
\end{itemize}

Where possible, this document will call out known deviations between the language as specified, and the current implementation in the open-source Slang compiler.

\Section{Document Conventions}{conventions}

The key words **must**, **must not**, **required**, **shall**, **shall not**, **should**, **should not**, **recommended**, **may**, and **optional** in this document are to be interpreted as described in [RFC 2119](https://www.ietf.org/rfc/rfc2119.txt).

The word "descriptive" in this document is to be interpreted as synonymous with "non-normative."

\SubSection{Typographical Conventions}{typographical}

This document makes use of the following conventions to improve readability:

* When a term is introduced and defined, it will be in italics. For example: a *duck* is something that looks like a duck and quacks like a duck.

* Code fragments in Slang or other programming languages use a fixed-width font. For example: `a + b`.

* References to language keywords are in bold. For example: **`if`**.

\SubSection{Callouts}{callouts}

This document uses a few kinds of *callouts*, which start bold word indicating the kind of callout. For example, the following is a note:

\begin{Note}
Notes are rendered like this.
\end{Note}

The kinds of callouts used in this document, are:

* **Example**s provide non-normative examples of how to use language constructs, or of their semantics.

* **Incomplete** callouts are non-normative markers of places where this document is lacking important information, or may be inaccurate.

* **Legacy** callouts are described below.

* **Limitation** callouts are used to provide non-normative discussion of how the Slang compiler implementation may deviate from the language as specified here.

* **Note**s give non-normative information that may help readers to understand and apply the normative information in the document.

* **Rationale** callouts provide non-normative explanations about *why* certain constructs, constraints, or rules are part of the language.

* **Syntax** callouts are described below.

* **Warning**s give non-normative information about language rules or programming practices that may lead to surprising or unwanted behavior.

\SubSection{Legacy Features}{legacy}

Some features of the Slang language are considered *legacy* features.
The language supports these constructs, syntax, etc. in order to facilitate compatibility with existing code in other GPU languages, such as HLSL.

**Legacy** callouts are used for the description of legacy features. These callouts are normative.

\SubSection{Grammar Productions}{grammar}

The syntax and lexical structure of the language is described **Syntax** callouts using a notation similar to EBNF.

* Nonterminals have names in camel case, and are rendered in italics. E.g., *CallExpression*.

* Terminals in the syntax, both punctiation and keywords are rendered in a fixed-width font. E.g., `func`.

* A question mark as a suffix indicates that the given element is optional. E.g., *Expression*?

* An asterisk as a suffix indicates that a given element may be repeated zero or more times. E.g., *Modifier* *

* A plus as a suffix indicates that a given element may be repeated one or more times. E.g., *AccessorDeclaration*+

* Parentheses in plain text (not fixed-width) are used for grouping. E.g., *Expression* (`,` *Expression*)*

> **Incomplete**: Also need conventions for how code points are named in productions in the lexical structure.

\Section{The Slang Standard Library}{stdlib}

Many constructs that appear to users of Slang as built-in syntax are instead defined as part of the *standard library* for Slang.
This document may, of neccessity, refer to constructs from the Slang standard library (e.g., the `Texture2D` type) that are not defined herein.

This document does not provide a normative definition of all of the types, functions, attributes, modifiers, capabilities, etc. that are defined in the Slang standard library.
The normative reference for standard-library constructs *not* defined in this document is the Slang Standard Library Reference.

> **Incomplete**: Need to turn that into a link.

In cases where a language construct is described in *both* this specification and the Standard Library Reference, the construct has all the capabilities and restrictions defined in either source.
In cases where the two sources disagree, there is a correctness issue in one or the other.

\Section{Normative References}{references}

> **Incomplete**: This section should list any external documents/specifications/etc. that are necessary in order to correctly interpret the information in this specification.