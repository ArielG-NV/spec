\Chapter{Parsing}{parse}

\SpecDef{Parsing} is the process of matching a sequence of tokens from the lexical grammar with a rule of the abstract syntax grammar.

\Section{Contexts}{context}

Parsing is always performed with respect to a \SpecDef{context}, which determines which identifiers are bound, and what they are bound to.

\begin{Syntax}
    contexts ::=
        \ContextVarA, \ContextVarB, ... \SynComment{context variable} \\
        \SynOr $null$ \SynComment{empty context} \\
        \SynOr \ContextVarA, \code{T} \SynComment{type} \\
        \SynOr \ContextVarA, \code{v : T} \SynComment{value} \\
        \SynOr \ContextVarA, \code{k} = \MetaVar{S} \SynComment{syntax rule}
\end{Syntax}

The \SpecDef{initial context} for parsing includes bindings for syntax rules corresponding to each declaration, statement, and expression keyword defined in this specification.

\begin{Example}
An \kw{if} expression begins with the \kw{if} keyword, so the initial context contains a binding for the identifier \lstinline{if} to a syntax rule that parses an \kw{if} expression. 
\end{Example}

The initial context also includes bindings for the declarations in the Slang standard library.
Implementations may include additional bindings in the initial context.



\Section{Strategies}{strat}

Parsing is always performed in one of two modes: unordered or ordered.
Unless otherwise specified, each grammar rule matches its sub-rules in the same mode.

\SubSection{Ordered}{ordered}

Parsing in \SpecDef{ordered} mode interleaves parsing and semantic analysis.
When the parser is in ordered mode and attempts to match the \SynRef{Declaration}, \SynRef{Statement}, or \SynRef{Expression} rules, and the lookahead is an \SynRef{Identifier}, it first looks up that identifier in the context.

\begin{itemize}
\item If the identifier is bound to a syntax rule \MetaVar{S}, then the parser only considers productions that can start with \MetaVar{S}.
\item Otherwise, it first matches the \SynRef{Term} rule, and then based on whether the result of checking that term is a type or a value, it either considers only productions starting with a \SynRef{Type} or those starting with an \SynRef{Expression}.
\end{itemize}

If the parser is in ordered mode and is about to match the \SynRef{DeclarationBody} rule, it switches to unordered mode before matching that rule, and switches back to ordered mode afterward.

\SubSection{Unordered}{unordered}

In \SpecDef{unordered} mode, semantic analysis is deferred.

When the parser is in unordered mode and attempts to match the \SynRef{Declaration} rule, and the lookahead is an \SynRef{Identifier}, it first looks up that identifier in the context.

\begin{itemize}
\item If the identifier is bound to a syntax rule \MetaVar{S}, then the parser only considers productions that can start with \MetaVar{S}.
\item Otherwise, it considers only productions that start with a \SynRef{TypeSpecifier}.
\end{itemize}
    
\begin{Syntax}
\SynDefine{BalancedTokens} \\
    \SynRef{BalancedToken}\SynStar

\SynDefine{BalancedToken} \\
    \lstinline|{| \SynRef{BalancedTokens} \lstinline|}|
    \SynOr \lstinline|(| \SynRef{BalancedTokens} \lstinline|)|
    \SynOr \lstinline|[| \SynRef{BalancedTokens} \lstinline|]|
    \SynOr \SynComment(any token other than \lstinline|{|, \lstinline|}|, \lstinline|(|, \lstinline|)|, \lstinline|[|, \lstinline|]|)
\end{Syntax}

A \SpecDef{balanced} rule in the grammar is one that meets one of the following criteria:
\begin{itemize}
\item starts with a \lstinline|{| and ends with a \lstinline|}|
\item starts with a \lstinline|(| and ends with a \lstinline|)|
\item starts with a \lstinline|[| and ends with a \lstinline|]|
\end{itemize}

Whenever the parser is in ordered mode and would attempt to match a balanced rule \MetaVar{R}, it instead matches the \SynRef{BalancedToken} rule and saves the token sequence that was matched.
Those tokens are then matched against \MetaVar{R} as part of semantic analysis, using the context that the checking rules specify.

\Section{Angle Brackets}{angle}

\SubSection{Opening Angle Brackets}{opening}

\SubSubSection{Ordered Mode}{ordered}

If the parser is in ordered mode with a lookahead of \code{<}, and it could match that token as part of either an \SynRef{InfixExpression} or a \SynRef{SpecializeExpression}, then it considers the term that has been matched so far:

\begin{itemize}
\item If the term has a generic type or overloaded type, then the \SynRef{SpecializeExpression} case is favored.
\item Otherwise the \SynRef{InfixExpression} case is favored.
\end{itemize}

\SubSubSection{Unordered Mode}{unordered}

If the parser is in ordered mode with a lookahead of \code{<}, and it could match that token as part of either an \SynRef{InfixExpression} or a \SynRef{SpecializeExpression}, then it first attempts to match the \SynRef{GenericArguments} rule. If that rule is matched successfully, then the new lookahead token is inspected:

\begin{itemize}
\item If the lookahead token is an \SynRef{Identifier}, \lstinline|(|, \SynRef{Literal}, or \SynRef{PrefixOperator}, then roll back to before the \code{<} was matched and favor the \SynRef{InfixExpression} rule.
\item Otherwise, use the generic arguments already matched as part of matching the \SynRef{SpecializeExpession} rule.
\end{itemize}

\SubSection{Closing Angle Brackets}{closing}

If the parser is attempting to match the closing \code{>} in the \SynRef{GenericParameters} or \SynRef{GenericArguments} rules, and the lookahead is any token other than \code{>} that starts with a greater-than character (\code{>=}, \code{>>}, or \code{>>=}), then it splits the opening \code{>} off and matches it, replacing the lookahead with a token comprised of the remaining characters (\code{=}, \code{>}, or \code{>=}).
