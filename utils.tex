% utils.tex
%
% This file defines all the utility commands that are needed to prepare the rest of the document.

\setcounter{secnumdepth}{6}


% \DefineVariable : a helper command for defining a getter and setter pair.
%
% Doing `\DefineVariable{Thing}` will allow you to use `\SetThing{...}` to
% set a variable and `\Thing` to access its current value.
%
%\makeatletter
%\newcommand*\DefineVariable[1]{\@namedef{Set#1}##1{\global\@namedef{#1}{##1}}}
%\makeatother
%
%\DefineVariable{ChapterLabel}
%\DefineVariable{SectionLabel}
%\DefineVariable{SubSectionLabel}
%
%\newcommand*{ChapterPath}{\ChapterLabel}
%\newcommand*{SectionPath}{\ChapterPath.\SectionLabel}
%\newcommand*{SubSectionPath}{\SectionPath.\SubSectionLabel}
%
%
%\newcommand*{\Chapter}[2]{\SetChapterLabel{#2}\chapter[#1\hfill[\ChapterPath]]\label{\ChapterPath}}
%\newcommand*{\Section}[2]{\SetSectionLabel{#2}\section[#1\hfill[\SectionPath]]\label{\SectionPath}}
%\newcommand*{\SubSection}[2]{\SetSubSectionLabel{#2}\subsection[#1\hfill[\SubSectionPath]]\label{\SubSectionPath}}
%
\newenvironment{Callout}[2]
    {
    \begin{center}
    \begin{tabular}{p{0.9\textwidth}}
    \cellcolor{#2}
    \textbf{#1}: \\
    }
    {
    \\\\
    \end{tabular}
    \end{center}
    }

%\newcommand{\DefineCallout}[3]{\newenvironment{#1}{\begin{Callout}{#2}{#3}}{\end{Callout}}}

\newcommand{\DefineCallout}[3]{
    \newtcolorbox{#1}[1][]{breakable,sharp corners, skin=enhancedmiddle jigsaw,parbox=false,
boxrule=0mm,leftrule=2mm,boxsep=0mm,arc=0mm,outer arc=0mm,attach title to upper,
after title={: \\}, coltitle=black,colback=gray!10,colframe=black, title={#2},
fonttitle=\bfseries,##1}}

\definecolor{NoteColor}{gray}{0.98}
\definecolor{IncompleteColor}{gray}{0.98}
\definecolor{SyntaxColor}{gray}{0.98}

\DefineCallout{Note}{Note}{NoteColor}
\DefineCallout{Incomplete}{Incomplete}{IncompleteColor}
\DefineCallout{SyntaxCallout}{Syntax}{SyntaxColor}
\DefineCallout{Lexical}{Syntax}{SyntaxColor}
\DefineCallout{Legacy}{Legacy}{SyntaxColor}
\DefineCallout{Rationale}{Rationale}{SyntaxColor}
\DefineCallout{Example}{Example}{SyntaxColor}
\DefineCallout{Semantics}{Semantics}{SyntaxColor}
\DefineCallout{Checking}{Checking}{SyntaxColor}
\DefineCallout{Description}{Description}{SyntaxColor}

%
%\definecolor{ListingKeywordColor}{gray}{0.5}
%\definecolor{ListingTypeColor}{gray}{0.5}
%
%\lstdefinelanguage{slang}{
%    classoffset=0,morekeywords={class,func,let,struct},\keywordstyle=\color{ListingKeywordColor},
%    classoffset=1,morekeywords={void,int,float,bool},\keywordstyle=\color{ListingTypeColor},
%    classoffset=0,
%    morecomment=[l]{//},
%    morecomment=[s]{/*}{*/},
%    morestring=[b]",
%}
%
%\lstnew

\newcommand*{\Chapter}[2]{\chapter{#1}}
\newcommand*{\Section}[2]{\section{#1}}
\newcommand*{\SubSection}[2]{\subsection{#1}}
\newcommand*{\SubSubSection}[2]{\subsubsection{#1}}
\newcommand*{\Paragraph}[2]{\paragraph{#1}}
\newcommand*{\SubParagraph}[2]{\subparagraph{#1}}



\definecolor{FREQUENCYCOLOR}{RGB}{0,176,80}
\definecolor{KEYWORDCOLOR}{RGB}{0,32,96}
\definecolor{COMMENTCOLOR}{RGB}{192,0,0}
\definecolor{TYPECOLOR}{RGB}{0,112,192}
\definecolor{KEYWORDARGCOLOR}{RGB}{144,155,170}

\definecolor{CodeBackgroundColor}{gray}{0.85}

\lstdefinestyle{SlangCodeStyle}{
    basicstyle=\mdseries\ttfamily,
    classoffset=0,
    morekeywords={if,else,struct,class,discard,while,for,do,func,let,var,property,break,continue,yield,interface,in,out,inout,ref,take,borrow},
    keywordstyle=\color{KEYWORDCOLOR}\bfseries\ttfamily,
    classoffset=1,
    morekeywords={@GroupShared},
    keywordstyle=\color{FREQUENCYCOLOR}\ttfamily,
    classoffset=2,
    morekeywords={void,int,float,Int,Float,Unit,bool,Bool},
    keywordstyle=\color{TYPECOLOR}\ttfamily,
    classoffset=0,
    moredelim=[is][\color{KEYWORDARGCOLOR}\ttfamily]{|}{|},
    moredelim=[is][\color{red}\ttfamily]{^}{^},
    morecomment=[l]//,
    commentstyle=\color{COMMENTCOLOR},
    tabsize=2,
    mathescape=true}

\lstset{
    style=SlangCodeStyle,
    basicstyle=\text\mdseries\ttfamily\small}

\lstnewenvironment{codeblock}[1][]
    {\lstset{
        style=SlangCodeStyle,
        basicstyle=\text\mdseries\ttfamily\small
        #1}}
    {}

\newcommand{\InlineCodeStyle}[1]{
    \fboxsep2pt
    \text{\ttfamily\mdseries\colorbox{CodeBackgroundColor}{#1}}
}

\newcommand{\code}[1]{\InlineCodeStyle{\lstinline[style=SlangCodeStyle]{#1}}}

%\newcommand{\code}[1]{\text{\colorbox{CodeBackgroundColor}{\lstinline[style=SlangCodeStyle]{#1}}}}
\newcommand{\kw}[1]{\text{\texttt{\textbf{\textcolor{KEYWORDCOLOR}{#1}}}}}

\newcommand{\lcurly}{\text{\InlineCodeStyle{\lstinline{\{}}}}
\newcommand{\rcurly}{\text{\InlineCodeStyle{\lstinline{\}}}}}

\newcommand{\SpecDefine}[1]{\text{\emph{#1}}}
\newcommand{\SpecDef}[1]{\text{\emph{#1}}}

\newcommand{\SynDefine}[1]{\text{\emph{#1} ::=}}
\newcommand{\SynRef}[1]{\text{\emph{#1}}}
\newcommand{\SynOr}{\text{ $\vert$ }}
\newcommand{\SynStar}{\text{*}}
\newcommand{\SynPlus}{\text{+}}
\newcommand{\SynOpt}{\text{?}}

\newcommand{\SynVar}[2][]{\text{\emph{#1:#2}}}


\newlist{SynEnvList}{itemize}{1}
\setlist[SynEnvList]{
  parsep=1ex, partopsep=0pt, itemsep=0pt, topsep=0pt, label={},
  leftmargin=2\leftmargin, listparindent=-\leftmargin,
  itemindent=\listparindent
}
\newenvironment{Syntax} {
    \begin{SyntaxCallout}
    \begin{SynEnvList}
    \item\relax
}{
    \end{SynEnvList}
    \end{SyntaxCallout}
}


\newcommand{\MetaDef}[1]{\textbf{$\mathbb{#1}$}}
\newcommand{\SynComment}[1]{\small{// #1}}

\newcommand{\MetaVar}[1]{\text{\textbf{$\mathbb{#1}$}}}

\newcommand{\ContextVarA}{\ensuremath{\Gamma_0}}
\newcommand{\ContextVarB}{\ensuremath{\Gamma_1}}
\newcommand{\ContextVarC}{\ensuremath{\Gamma_2}}
\newcommand{\ContextVarD}{\ensuremath{\Gamma_3}}
\newcommand{\ContextVarE}{\ensuremath{\Gamma_4}}
\newcommand{\ContextVarF}{\ensuremath{\Gamma_5}}

\newcommand{\ExprVarE}{\ensuremath{e}}
\newcommand{\ExprVarF}{\ensuremath{f}}

\newcommand{\StmtVarS}{\ensuremath{s}}
\newcommand{\StmtVarT}{\ensuremath{t}}

\newcommand{\TypeVarT}{\ensuremath{T}}
\newcommand{\TypeVarU}{\ensuremath{U}}

\newcommand{\BindingVarX}{\ensuremath{\alpha}}
\newcommand{\BindingVarY}{\ensuremath{\beta}}

\newcommand{\Char}[1]{\code{#1}}

\newcommand{\ResultType}{\text{\textsc{Return}}}
\newcommand{\BreakLabel}{\text{\textsc{Break}}}
\newcommand{\ContinueLabel}{\text{\textsc{Continue}}}

\newcommand{\RequireCap}[1]{\text{\textsc{Require} \textsc{#1}}}


\newcommand{\DerivationRule}[3][]{\ensuremath{\trfrac[#1]{#2}{#3}}}

%
% Utilities for use in `Checking` callouts:
%

\newcommand{\SynthExpr}[4]{\ensuremath{#1 \vdash #2 \Rightarrow #3 \vdash #4}}
\newcommand{\CheckExpr}[4]{\ensuremath{#1 \vdash #2 \Leftarrow #3 \vdash #4}}

\newcommand{\CheckType}[3]{\ensuremath{#1 \vdash \textsc{type}\ #2\ \textsc{ok} \vdash #3}}

\newcommand{\CheckStmt}[3]{\ensuremath{#1 \vdash \textsc{stmt}\ #2\ \textsc{ok} \vdash #3}}
\newcommand{\CheckStmts}[3]{\ensuremath{#1 \vdash \textsc{stmts}\ #2\ \textsc{ok} \vdash #3}}
\newcommand{\CheckDecl}[3]{\ensuremath{#1 \vdash \textsc{decl}\ #2\ \textsc{ok} \vdash #3}}

\newcommand{\CheckCases}[3]{\ensuremath{#1 \vdash \textsc{CheckAlternatives}(#2, #3)}}
\newcommand{\CheckCase}[3]{\ensuremath{#1 \vdash \textsc{CheckAlternative}(#2, #3)}}
\newcommand{\CheckCaseLabel}[3]{\ensuremath{#1 \vdash \textsc{CheckAlternativeLabel}(#2, #3)}}

% Check that the given context contains an entry of the given form.
\newcommand{\ContextContains}[2]{\ensuremath{#1 \vdash #2}}

% Check that lookup up the given symbol in the context yields the given value/type/whatever.
\newcommand{\ContextLookup}[3]{\ensuremath{#1 ( #2 ) = #3}}

% Check that a type conforms to a given interface in the input context.
% Yields an output context (which may include additional constraints).
\newcommand{\CheckConforms}[4]{\ensuremath{#1 \vdash #2\ \textsc{ConformsTo} #3 \vdash #4}}


% Check that an attempt to construct an instance of the given type
% with the given argument list is valid.
\newcommand{\CheckConstruct}[4]{\ensuremath{#1 \vdash \textsc{Construct}(#2, #3) \vdash #4}}

% Check/synthesize a call to an expression with arguments.
\newcommand{\CheckCall}[5]{\ensuremath{#1 \vdash \textsc{Call}(#2, #3) \Leftarrow #4 \vdash #5}}
\newcommand{\SynthCall}[5]{\ensuremath{#1 \vdash \textsc{Call}(#2, #3) \Rightarrow #4 \vdash #5}}

% Check/synthesize lookup of a name
\newcommand{\CheckLookup}[4]{\ensuremath{#1 \vdash \textsc{Lookup}(#2) \Leftarrow #3 \vdash #4}}
\newcommand{\SynthLookup}[3]{\ensuremath{#1 \vdash \textsc{Lookup}(#2) \Rightarrow #3}}
